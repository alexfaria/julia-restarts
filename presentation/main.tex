\documentclass{beamer}
%
% Choose how your presentation looks.
%
% For more themes, color themes and font themes, see:
% http://deic.uab.es/~iblanes/beamer_gallery/index_by_theme.html
%
\mode<presentation>
{
  % \usetheme{boxes}        % or try Darmstadt, Madrid, Warsaw, ...
  \usetheme{CambridgeUS}
  \usecolortheme{dolphin} % or try albatross, beaver, crane, ...
  \usefonttheme{structurebold}  % or try serif, structurebold, ...
  % \setbeamertemplate{navigation symbols}{}
  \setbeamertemplate{caption}[numbered]
} 

\usepackage[english]{babel}
\usepackage[utf8]{inputenc}
\usepackage{inconsolata}
\usepackage[T1]{fontenc}
\usepackage{listings}
\usepackage{fancyvrb}
\usepackage{minted}

\usepackage{verbatim}
\makeatletter
\def\verbatim@font{\scriptsize\ttfamily}
\makeatother

% https://tex.stackexchange.com/questions/460731/highlight-color-a-part-of-text-in-block-in-beamer
\usepackage{soul}
\makeatletter
\let\HL\hl
\renewcommand\hl{%
  \let\set@color\beamerorig@set@color
  \let\reset@color\beamerorig@reset@color
  \HL}
\makeatother

\usepackage{jlcode}

% basic font style

\def\jlbasicfont{\ttfamily\tiny\selectfont}

\lstdefinestyle{jlcodecolorstyle}{%
    basicstyle={\loadcolors\color{jlstring}\jlbasicfont},
    keywordstyle={[1]\color{jlkeyword}\bfseries},
    keywordstyle={[2]\color{jlliteral}},
    keywordstyle={[3]\color{jlbuiltin}},
    keywordstyle={[4]\color{jlmacros}},
    commentstyle={\color{jlcomment}},
    stringstyle={\color{jlstring}},
    identifierstyle={\color{jlbase}}
}

\title[Julia Restarts]{Beyond Exception Handling}
\subtitle{Implementing restarts in Julia}
% \author{Alexandre Faria 97005 \and Rafael Pinto 97153}
%\institute{MEIC - PA 2019/2020}
% \date{Date of Presentation}

\author[Alexandre \& Rafael]{%
  \texorpdfstring{%
    \begin{columns}
      \column{.5\linewidth}
      \centering
      Alexandre Faria \\ 97005
      \column{.5\linewidth}
      \centering
      Rafael Pinto \\ 97153
    \end{columns}
 }
 {Alexandre Faria, Rafael Pinto}
}
\institute[IST]{Instituto Superior Técnico}
\date[May 2020]{May 2020}


\begin{document}
\begin{frame}
  \titlepage
\end{frame}

% Uncomment these lines for an automatically generated outline.
%\begin{frame}{Outline}
%  \tableofcontents
%\end{frame}


%%%%%%%%%%%%%%%%%%%%%%%%%%%%%%%%% SLIDE %%%%%%%%%%%%%%%%%%%%%%%%%%%%%%%%% 
\section{Data Structures}
\begin{frame}[fragile,t]{Data Structures and Global Variables}

% \vskip 1cm

\begin{itemize}
  \item ReturnFromException is used to transfer control between return\_from to block
  \item Handlers are stored in a global stack
  \item A counter is used to keep block names unique
  \item Group all the handlers from a handler\_bind form and store those groups as a list in current\_available\_handlers
\end{itemize}

\vfill

\begin{lstlisting}[language=julia, style=jlcodestyle]
# exception used in block and return_from
struct ReturnFromException <: Exception
    name    # block name
    value   # return value
end

# global variables
current_available_restarts = []     # stack of restarts
current_available_handlers = []     # stack of handler groups
current_block_id = 0                # block name counter
\end{lstlisting}
\end{frame}


%%%%%%%%%%%%%%%%%%%%%%%%%%%%%%%%% SLIDE %%%%%%%%%%%%%%%%%%%%%%%%%%%%%%%%% 
\section{Handler Binding}

\begin{frame}[fragile,t]{Handler Bind}

\vfill

\begin{enumerate}
    \item Push handlers onto the start of the current\_available\_handlers stack
    \item Call the function
    \item Pop the handlers from the stack
\end{enumerate}

\vfill

\begin{lstlisting}[language=julia, style=jlcodestyle]
function handler_bind(func, handlers...)
    global current_available_handlers

    pushfirst!(current_available_handlers, handlers)

    try
        func()
    finally
        popfirst!(current_available_handlers)
    end
end
\end{lstlisting}

\end{frame}

%%%%%%%%%%%%%%%%%%%%%%%%%%%%%%%%% SLIDE %%%%%%%%%%%%%%%%%%%%%%%%%%%%%%%%% 
\section{Error Handling}

\begin{frame}[fragile,t]{Error Handling}

\vfill

\begin{itemize}
    \item The handler corresponding to the first matched exception is executed
    \item Otherwise, the error is not caught and is thrown normally
\end{itemize}

\vfill

\begin{lstlisting}[language=julia, style=jlcodestyle]
function error(exception::Exception)
    for handler_group in current_available_handlers
        for handler in handler_group
            if exception isa handler[1]
                handler[2](exception)
                break
            end
        end
    end
    throw(exception)
end
\end{lstlisting}

\end{frame}

%%%%%%%%%%%%%%%%%%%%%%%%%%%%%%%%% SLIDE %%%%%%%%%%%%%%%%%%%%%%%%%%%%%%%%% 
\section{Block \& Return From}

\begin{frame}[fragile,t]{Block}
\begin{itemize}
  \item Use the name in the exception to know which block is supposed to handle it
  \item If this block doesnt match the name, propagate the exception upwards in the chain call to another block
\end{itemize}
\vfill
\begin{lstlisting}[language=julia, style=jlcodestyle]
function block(func)
    global current_block_id
    current_block_id += 1
    name = current_block_id
    try
        func(name)
    catch e
        if e isa ReturnFromException && e.name == name
            return e.value
        end
        rethrow(e)
    end
end
\end{lstlisting}
\end{frame}

%%%%%%%%%%%%%%%%%%%%%%%%%%%%%%%%% SLIDE %%%%%%%%%%%%%%%%%%%%%%%%%%%%%%%%% 

\begin{frame}[fragile,t]{Return From}
\begin{itemize}
  \item Throw a ReturnFromException exception with the block name and the value to return
  \item The block form will use this information to gain the control flow and return 
\end{itemize}
\vfill
\begin{lstlisting}[language=julia, style=jlcodestyle]
function return_from(name, value = nothing)
    throw(ReturnFromException(name, value))
end
\end{lstlisting}
\end{frame}

%%%%%%%%%%%%%%%%%%%%%%%%%%%%%%%%% SLIDE %%%%%%%%%%%%%%%%%%%%%%%%%%%%%%%%%
\section{Restart Binding}
\begin{frame}[fragile,t]{Restart Binding}
    
\begin{itemize}
    \item Before calling the function, we add each restart to the stack so that it's available later to be invoked
    \item The context of execution is inserted into a block form and the name 'rb\_block' is passed into the restart function by wrapping it in an anonymous functions that capturesthe value of rb\_block
\end{itemize}
\vfill
\begin{lstlisting}[language=julia, style=jlcodestyle]
function restart_bind(func, restarts...)
    block() do rb_block
        for r in restarts
            pushfirst!(current_available_restarts,
                (r[1] => (args...) -> return_from(rb_block, r[2](args...))))
        end
        ...
    end
end
\end{lstlisting}
\end{frame}

%%%%%%%%%%%%%%%%%%%%%%%%%%%%%%%%% SLIDE %%%%%%%%%%%%%%%%%%%%%%%%%%%%%%%%%
\begin{frame}[fragile,t]{Restart Binding}
\begin{itemize}
    \item After establishing the restarts, the execution can then be passed to the given function
    \item Once the block gets the control of the execution back, the restarts established for that block must be removed
    \item The finally block guarantees that the restarts are always removed from the stack, even if an exceptional situation occurs
\end{itemize}
\begin{lstlisting}[language=julia, style=jlcodestyle]
function restart_bind(func, restarts...)
    block() do rb_block
        ...
        try
            func()
        finally
            for i in restarts
                popfirst!(current_available_restarts)
            end
        ...
\end{lstlisting}
\end{frame}

%%%%%%%%%%%%%%%%%%%%%%%%%%%%%%%%% SLIDE %%%%%%%%%%%%%%%%%%%%%%%%%%%%%%%%%
\section{Restarts}
\begin{frame}[fragile,t]{Available Restarts}
\begin{itemize}
    \item The ability to know whether a restart is available is useful to conditionally invoke a restart
    \item The function looks for any restart, with the given name, in the current available restarts
    \item The current available restarts contains the restarts that were established along the call chain
\end{itemize}
\vfill
\begin{lstlisting}[language=julia, style=jlcodestyle]
function available_restart(name)
    global current_available_restarts
    any(r -> r[1] == name, current_available_restarts)
end
\end{lstlisting}
\end{frame}

%%%%%%%%%%%%%%%%%%%%%%%%%%%%%%%%% SLIDE %%%%%%%%%%%%%%%%%%%%%%%%%%%%%%%%%

\begin{frame}[fragile,t]{Invoking Restarts}

\begin{itemize}
    \item In order to invoke a restart, the program evaluates if there are currently available restarts that match the name provided by the handler
    \item If so, it will invoke the restart with the given arguments which will resume the execution in some pre-established point
\end{itemize}
\vfill
\begin{lstlisting}[language=julia, style=jlcodestyle]
function invoke_restart(name, args...)
    global current_available_restarts
    i = findfirst(r -> r[1] == name, current_available_restarts)
    func = current_available_restarts[i][2]
    func(args...)
end
\end{lstlisting}
\end{frame}

%%%%%%%%%%%%%%%%%%%%%%%%%%%%%%%%% SLIDE %%%%%%%%%%%%%%%%%%%%%%%%%%%%%%%%%

\section{Extensions}
\begin{frame}[fragile,t]{Signal Extension}

\begin{itemize}
    \item The signal function is simply the error function except it doesn't throw the exception if there are no available handlers
\end{itemize}
\vfill
\begin{lstlisting}[language=julia, style=jlcodestyle]
function signal(exception::Exception)
    global current_available_handlers

    for handler_group in current_available_handlers
        for handler in handler_group
            if exception isa handler[1]
                handler[2](exception)
                break
            end
        end
    end
end
\end{lstlisting}
\end{frame}

\begin{frame}[fragile,t]{Handler Case Macro}

\begin{itemize}
    \item This macro wraps the handler in a block and return\_from form
    %TODO: dizer aqui mais
\end{itemize}
\vfill
\begin{lstlisting}[language=julia, style=jlcodestyle,]
macro handler_case(func, handlers...)
    let
        escape_block = Symbol("escape_block")
        handler_func = func

        for handler in handlers
            handler_body = handler.args[3].args[2]
            handler.args[3].args[2] = :(return_from($(escape_block), $handler_body))
        end

# hack in order to work with two macro calling syntax
# @handler_case(reciprocal(0), DivisionByZero => (c) -> println("zero!"))
# @handler_case(DivisionByZero => (c) -> print("zero!")) do reciprocal(0) end
        if func.head == :call
            handler_func = :(() -> begin $func end)
        end
        quote
            block() do $(escape_block)
                handler_bind($handler_func, $(handlers...))
            end
        end
    end
end

\end{lstlisting}
\end{frame}


\begin{frame}[fragile,t]{Restart Case Macro}

\begin{itemize}
    \item This macro simply calls restart\_bind since it's already in the same form as restart\_case
\end{itemize}
\vfill
\begin{lstlisting}[language=julia, style=jlcodestyle,]
macro restart_case(func, restarts...)
    let
        quote
            restart_bind($func, $(restarts...))
        end
    end
end
\end{lstlisting}
\end{frame}


\section{Interactive Restarts}
\begin{frame}[fragile,t]{Interactively Invoking a Restart}

\begin{itemize}
    \item Example of the interactive restarts extensions
\end{itemize}
\vfill

\begin{lstlisting}[language=julia, style=jlcodestyle,]
struct DivisionByZero <: Exception end

reciprocal(value) =
    restart_bind(
        Restart(:return_zero => (args...) -> 0, report="Return Zero"),
        Restart(:return_value => identity, report="Return Value",
                                           interactive=true,
                                           test = () -> false),
        Restart(:retry_using => reciprocal, report="Retry with another value",
                                            interactive=true)
    ) do
        value == 0 ? error(DivisionByZero()) : 1 / value
    end
\end{lstlisting}

\end{frame}

\begin{frame}[fragile,t]{Interactively Invoking a Restart}

\begin{itemize}
    \item Only the functions which 'test' returns true are shown
    \item The user handling of restarts is only run when no handlers execute a non-local transfer control
\end{itemize}
\vfill

\begin{lstlisting}[language=julia, style=jlcodestyle,]
julia> reciprocal(0)

#<DivisionByZero()>#
 [Condition of type DivisionByZero]
Restarts:
 1: [RETRY_USING] Retry with another parameter
 2: [RETURN_ZERO] Return Zero
Pick: 2
0
\end{lstlisting}
\end{frame}
\begin{frame}[fragile,t]{Interactively Invoking a Restart}

\begin{itemize}
    \item Only the functions which 'test' returns true are shown
    \item There's also the possibility to ask the user for input by setting 'interactive' to true
    \item The 'interactive' variable was simplified to be easier to implement interactive restarting
\end{itemize}
\vfill

\begin{lstlisting}[language=julia, style=jlcodestyle,]
julia> reciprocal(0)

#<DivisionByZero()>#
 [Condition of type DivisionByZero]
Restarts:
 1: [RETRY_USING] Retry with another parameter
 2: [RETURN_ZERO] Return Zero
Pick: 1
Input: 10
0.1
\end{lstlisting}
\end{frame}

\begin{frame}[fragile,t]{Interactively Invoking a Restart}

\begin{itemize}
    \item Limited backwards compatibility with normal restarts
    \item The interactive handling also works with normal restarts but no input is asked from the user, therefore it only works with restarts which take no parameters
\end{itemize}
\vfill

\begin{lstlisting}[language=julia, style=jlcodestyle,]
julia> reciprocal(0)

#<DivisionByZero()>#
 [Condition of type DivisionByZero]
Restarts:
 1: [RETRY_USING] retry_using
 2: [RETURN_VALUE] return_value
 3: [RETURN_ZERO] return_zero
Pick: 3
0
\end{lstlisting}
\end{frame}

\begin{frame}[fragile,t]{Restart Struct}

\begin{itemize}
    \item This structure holds the values for an Extended Restart
    \item This is used to be able to print the restart with a name, test if its available, and get input from the user
\end{itemize}
\vfill
\begin{lstlisting}[language=julia, style=jlcodestyle,]
mutable struct Restart
    restart     # restart pair :return_zero => () -> 0
    test        # test if the restart is available
    interactive # simplified to be simply a boolean
                # instead of a custom function that asks the user for parameters
    report      # name of the restart that appears in the prompt

    function Restart(restart; test= ()->true, interactive=false, report=nothing)
        report = report == nothing ? string(name) : report
        new(restart, test, interactive, report)
    end
end
\end{lstlisting}
\end{frame}

\begin{frame}[fragile,t]{Interactively Invoking a Restart}

\begin{itemize}
    \item This method is added to the global stack of handlers so that its the last to be reached
    \item This method is only called if no handlers execute a non-local transfer of control
    \item It searches for restarts in the global stack and prints them
\end{itemize}
\vfill

\begin{lstlisting}[language=julia, style=jlcodestyle,]
current_available_handlers = push!(current_available_handlers,
                [Exception => (c) -> picking_interactive_restart_handler(c)])
function picking_interactive_restart_handler(exception)
    global current_available_restarts
    restarts = filter(r -> (!(r[2].r isa Restart) ||
                             (r[2].r isa Restart && r[2].r.test())),
                        current_available_restarts)
    if length(restarts) > 0
        println()
        println("#<$(exception)>#")
        println(" [Condition of type $(typeof(exception))]")
        println("Restarts:")
        ...
    end
end
\end{lstlisting}

\end{frame}
\begin{frame}[fragile,t]{Interactively Invoking a Restart}

\begin{itemize}
    \item The 'if' inside the for loop makes it 'backwards' compatible with non-extended restarts (:name => func)
\end{itemize}
\vfill

\begin{lstlisting}[language=julia, style=jlcodestyle,]
current_available_handlers = push!(current_available_handlers, [Exception => (c) -> picking_interactive_restart_handler(c)])
function picking_interactive_restart_handler(exception)
    ...
    if length(restarts) > 0
        ...
        for r in restarts
            if r[2].r isa Restart
                println(" $(i): [$(uppercase(String(r[1])))] $(r[2].r.report)")
            else
                println(" $(i): [$(uppercase(String(r[1])))] $(r[1])")
            end
            i += 1
        end
        print("Pick: ")
        restart = readline()
        i = tryparse(Int, restart)
        restart = restarts[i]
        ...
end
\end{lstlisting}
\end{frame}


\begin{frame}[fragile,t]{Interactively Invoking a Restart}

\begin{itemize}
    \item It then selects the restart and uses invoke\_restart
    \item Passing parameters to non-extended restarts was not implemented, nor was any type of type checking for parameters which would increase the complexity of this extension
\end{itemize}
\vfill

\begin{lstlisting}[language=julia, style=jlcodestyle,]
function picking_interactive_restart_handler(exception)
    ...
    if length(restarts) > 0
        ...
        value = nothing
        if restart[2].r isa Restart && restart[2].r.interactive
            print("Input: ")
            value = Meta.parse(readline())
            invoke_restart(restart[1], value)
        else
            invoke_restart(restart[1])
        end
    end
end
\end{lstlisting}
\end{frame}

\begin{frame}[fragile,t]{Interactively Invoking a Restart}

\begin{itemize}
    \item In order to support this, restart\_bind was also extended by adding another generic function which receives the new extended restarts
    \item This was necessary in order to convert them into the normal, un-extended, pair format (:name => func)
\end{itemize}
\vfill

\begin{lstlisting}[language=julia, style=jlcodestyle,]
function restart_bind(func, restarts::Restart...)
    global current_available_restarts
    block() do rb_block
        for r in restarts
            # wrap inside an anonymous function that
            # captures the value of rb_block
            pushfirst!(current_available_restarts,
               (r.restart[1] => (args...) -> return_from(rb_block,
                                                    r.restart[2](args...))))
        end
        try
            func()
        finally
            for i in restarts
                popfirst!(current_available_restarts)
            end
end end end
\end{lstlisting}
\end{frame}

\end{document}
